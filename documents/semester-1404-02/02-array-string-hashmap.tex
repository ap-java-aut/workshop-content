%%%%%%%%%%%%%%%%%%%%%%%%%%%%%%%%%%%%%%%%%%%%%%%%%%%%%%%%%%%%
% (01) Input Settings
%%%%%%%%%%%%%%%%%%%%%%%%%%%%%%%%%%%%%%%%%%%%%%%%%%%%%%%%%%%%
\documentclass[aspectratio=169,xcolor={dvipsnames}]{beamer}

%%%%%%%%%%%%%%%%%%%%%%%%%%%%%%%%%%%%%%%%%%%%%%%%%%%%%%%%%%%%
% (1) LOAD PACKAGES *BEFORE* xepersian (bidi)
%%%%%%%%%%%%%%%%%%%%%%%%%%%%%%%%%%%%%%%%%%%%%%%%%%%%%%%%%%%%
\usepackage{amsmath}
\allowdisplaybreaks
\usepackage{xcolor}
\usepackage{graphicx}
\usepackage{float}
\usepackage[colorlinks=true]{hyperref}
\usepackage{url}
\usepackage{minted}
\usepackage{tcolorbox}
\tcbuselibrary{breakable,skins,minted}
\usepackage{amsfonts}
\usepackage{cancel}
\usepackage{tikz}
\usetikzlibrary{shapes.geometric}
\usepackage{caption}
\usetikzlibrary{arrows.meta,calc,decorations.pathreplacing,positioning}
\usepackage{nicematrix}
\usetikzlibrary{plotmarks}
\usepackage{xstring}
\usepackage{longtable}
\usepackage{booktabs}
\usepackage{array}

%%%%%%%%%%%%%%%%%%%%%%%%%%%%%%%%%%%%%%%%%%%%%%%%%%%%%%%%%%%%
% (2) NOW load xepersian (which will load bidi)
%%%%%%%%%%%%%%%%%%%%%%%%%%%%%%%%%%%%%%%%%%%%%%%%%%%%%%%%%%%%
\usepackage{xepersian}

%%%%%%%%%%%%%%%%%%%%%%%%%%%%%%%%%%%%%%%%%%%%%%%%%%%%%%%%%%%%
% (3) FONT SETUP (after xepersian)
%%%%%%%%%%%%%%%%%%%%%%%%%%%%%%%%%%%%%%%%%%%%%%%%%%%%%%%%%%%%
\settextfont[
Path = fonts/,
Extension = .ttf,
UprightFont = Estedad-Regular,
BoldFont = Estedad-Bold,
ItalicFont = Estedad-Regular,
BoldItalicFont = Estedad-Bold,
Scale = 1.0
]{Estedad}

\newfontfamily\estedadThin{Estedad-Thin.ttf}
\newfontfamily\estedadLight{Estedad-Light.ttf}
\newfontfamily\estedadMedium{Estedad-Medium.ttf}
\newfontfamily\estedadSemiBold{Estedad-SemiBold.ttf}
\newfontfamily\estedadExtraBold{Estedad-ExtraBold.ttf}
\newfontfamily\estedadBlack{Estedad-Black.ttf}
\newfontfamily\estedadBold{Estedad-Bold.ttf}

\defaultfontfeatures[\rmfamily,\sffamily]{}
\setlatintextfont[
Ligatures      = TeX,
UprightFont    = * ,
BoldFont       = * Bold ,
ItalicFont     = * Italic ,
BoldItalicFont = * Bold Italic
]{Times New Roman}

%%%%%%%%%%%%%%%%%%%%%%%%%%%%%%%%%%%%%%%%%%%%%%%%%%%%%%%%%%%%
% (4) BEAMER THEME CONFIGURATION
%%%%%%%%%%%%%%%%%%%%%%%%%%%%%%%%%%%%%%%%%%%%%%%%%%%%%%%%%%%%
\usetheme{Madrid}
\usecolortheme{default}

% Customize colors
\definecolor{primarycolor}{RGB}{0,51,102}
\definecolor{secondarycolor}{RGB}{0,102,204}
\definecolor{accentcolor}{RGB}{204,0,0}

\setbeamercolor{structure}{fg=primarycolor}
\setbeamercolor{frametitle}{bg=primarycolor,fg=white}
\setbeamercolor{title}{fg=primarycolor}
\setbeamercolor{subtitle}{fg=secondarycolor}
\setbeamercolor{author}{fg=black}
\setbeamercolor{date}{fg=black}
\setbeamercolor{institute}{fg=black}

% Remove navigation symbols
\setbeamertemplate{navigation symbols}{}

% Add frame numbers
\setbeamertemplate{footline}[frame number]

% Customize frame title
\setbeamertemplate{frametitle}{
  \vspace{0.5cm}
  \begin{beamercolorbox}[wd=\paperwidth,ht=1.2cm,dp=0.5cm,left]{frametitle}
    \hspace{0.5cm}\insertframetitle
  \end{beamercolorbox}
}

%%%%%%%%%%%%%%%%%%%%%%%%%%%%%%%%%%%%%%%%%%%%%%%%%%%%%%%%%%%%
% (5) CUSTOM CODE BLOCK COMMAND - PERSIAN COMPATIBLE
%%%%%%%%%%%%%%%%%%%%%%%%%%%%%%%%%%%%%%%%%%%%%%%%%%%%%%%%%%%%
\definecolor{darkcodebg}{rgb}{0.1,0.1,0.1}
\definecolor{darkcodefg}{rgb}{0.95,0.95,0.95}
\definecolor{linenumcolor}{rgb}{0.7,0.7,0.7}
\definecolor{codebg}{rgb}{0.95,0.95,0.95}

\setminted{
	style       = vs,
	bgcolor     = codebg,
	tabsize     = 4,
	tab         = {~~~~},
}

\renewcommand{\theFancyVerbLine}{\textcolor{linenumcolor}{\small\arabic{FancyVerbLine}}}

\NewDocumentEnvironment{codebox}{O{} m}{%
	\VerbatimEnvironment%
	\setLTR%
	\latinfont%
	\color{darkcodefg}%
	\begin{minted}[
		style=monokai,
		bgcolor=darkcodebg,
		fontsize=\footnotesize,
		fontfamily=tt,
		breaklines,
		breakanywhere=false,
		tabsize=4,
		showtabs=false,
		showspaces=false,
		baselinestretch=1.0,
		linenos,
		numbersep=5pt,
		numbers=left,
		stepnumber=1,
		firstnumber=1,
		xleftmargin=0pt,
		frame=none,
		stripnl=false,
		encoding=utf8,
		]{#2}%
	}{%
	\end{minted}%
}

%%%%%%%%%%%%%%%%%%%%%%%%%%%%%%%%%%%%%%%%%%%%%%%%%%%%%%%%%%%%
% (6) Fix PANDOC DEFAULTS
%%%%%%%%%%%%%%%%%%%%%%%%%%%%%%%%%%%%%%%%%%%%%%%%%%%%%%%%%%%%
% Prevent pandocbounded wrapper for images
\makeatletter
\def\pandocbounded#1{#1}
\makeatother

% Define tightlist as empty (no extra spacing in lists)
\providecommand{\tightlist}{}

%%%%%%%%%%%%%%%%%%%%%%%%%%%%%%%%%%%%%%%%%%%%%%%%%%%%%%%%%%%%
% (7) PRESENTATION NOTES SUPPORT
%%%%%%%%%%%%%%%%%%%%%%%%%%%%%%%%%%%%%%%%%%%%%%%%%%%%%%%%%%%%
% Enable notes on second screen for manual addition later
% Uncomment the line below when you want to see notes during presentation
% \setbeameroption{show notes on second screen}

% For now, notes are hidden but can be manually added to slides
\setbeameroption{hide notes}


%%%%%%%%%%%%%%%%%%%%%%%%%%%%%%%%%%%%%%%%%%%%%%%%%%%%%%%%%%%%
% (02) DOCUMENT INFORMATION - FROM YAML METADATA
%%%%%%%%%%%%%%%%%%%%%%%%%%%%%%%%%%%%%%%%%%%%%%%%%%%%%%%%%%%%
\newcommand{\DocTitle}{آرایه و ArrayList ، کلاس String و HashMap}

\newcommand{\CourseTitle}{برنامه‌سازی پیشرفته و کارگاه}

\newcommand{\Semester}{ترم دوم ۱۴۰۴-۱۴۰۵}

\newcommand{\FirstProfessor}{دکتر روح‌الله احمدیان}

\newcommand{\SecondProfessor}{}

\newcommand{\Writers}{پریا قره‌ خانی، نگین موسی‌ یی جو، امیرمحمد
ذاکر، کیانا پهلوان}

\newcommand{\Faculty}{دانشکده ریاضی و علوم کامپیوتر}

\newcommand{\University}{دانشگاه صنعتی امیرکبیر}

%%%%%%%%%%%%%%%%%%%%%%%%%%%%%%%%%%%%%%%%%%%%%%%%%%%%%%%%%%%%
% DOCUMENT BODY
%%%%%%%%%%%%%%%%%%%%%%%%%%%%%%%%%%%%%%%%%%%%%%%%%%%%%%%%%%%%
\begin{document}
	
%%%%%%%%%%%%%%%%%%%%%%%%%%%%%%%%%%%%%%%%%%%%%%%%%%%%%%%%%%%%
% === TITLE PAGE ===
%%%%%%%%%%%%%%%%%%%%%%%%%%%%%%%%%%%%%%%%%%%%%%%%%%%%%%%%%%%%
\begin{titlepage}
	\centering
	\vspace*{1cm}
	
	%— Smaller logo (20% of text width)
	\begin{figure}[H]
		\centering
		\includegraphics[width=0.20\textwidth]{images/logo-fa.png}
	\end{figure}
	
	\vspace{1cm}
	
	{\fontsize{18}{28}\selectfont \textbf{\CourseTitle} }
	
	\vspace{3cm}
	
	{\fontsize{22}{28}\selectfont \textbf{\DocTitle} }
	
	\vspace{3cm}
	
	{\fontsize{14}{22}\selectfont \rl{نگارش:}}\\
	\vspace{0.5cm}
	{\fontsize{16}{22}\selectfont \textbf{ \Writers } }

	
	\vspace{1cm}
	
	{\fontsize{14}{22}\selectfont \rl{استاد درس:}}\\
	\vspace{0.5cm}
	{\fontsize{16}{22}\selectfont \textbf{ \FirstProfessor } }
	
	
	\vfill
	
	{\fontsize{14}{18}\selectfont \textbf{\Faculty} }\\
	\vspace{0.5cm}
	{\fontsize{14}{18}\selectfont \textbf{\University} }
	\vspace{0.5cm}
	
	\par
	
	{\fontsize{14}{18}\selectfont \textbf{\Semester} }
	
\end{titlepage}

%— Switch to our fancy header/footer from this page onward
\clearpage

%%%%%%%%%%%%%%%%%%%%%%%%%%%%%%%%%%%%%%%%%%%%%%%%%%%%%%%%%%%%
% === TABLE OF CONTENTS ===
%%%%%%%%%%%%%%%%%%%%%%%%%%%%%%%%%%%%%%%%%%%%%%%%%%%%%%%%%%%%
\tableofcontents
\clearpage
\pagestyle{aftertitle}

%%%%%%%%%%%%%%%%%%%%%%%%%%%%%%%%%%%%%%%%%%%%%%%%%%%%%%%%%%%%
% === MAIN CONTENT ===
%%%%%%%%%%%%%%%%%%%%%%%%%%%%%%%%%%%%%%%%%%%%%%%%%%%%%%%%%%%%
\chapter{\DocTitle}
\section{مقدمه}

این داکیومنت برای آشنایی با چند ساختمان داده \lr{(Data Structure)} آشنا
در زبان جاوا می‌باشد، قرار است یاد بگیریم که این ساختمان داده ها به طور
کلی به چه نحوی عمل می‌کنند و چه توابعی برای کار کردن با آن‌ها وجود دارد.

\section{آرایه - \lr{Array}}

اگر بخواهید یک سری داده هم‌نوع را ذخیره کنید، می‌توانید برای هر کدام به
طور جداگانه یک متغیر درست کنید و مقدار دهی کنید، اما این روش خیلی بهینه
نیست! راه بهتر این است که از آرایه استفاده کنید. آرایه یک ساختار داده‌است
که می‌تواند چندین مقدار هم‌نوع را داخل یک متغیر نگه دارد و مدیریت این
داده‌ها را خیلی راحت‌تر می‌کند. شما مشابه این \lr{Data Structure} را در
زبان \lr{C} هم دیدید.

بیایید ببینیم آرایه چه ویژگی‌هایی دارد و چطور کار می‌کند:

\begin{itemize}
\tightlist
\item
  \textbf{اندیس گذاری \lr{(Indexing)}:} همان‌طور که احتمالا از زبان
  \lr{C} به یاد دارید، آرایه‌ها \lr{zero-based} هستند، یعنی اولین عنصر
  آن‌ها اندیس ۰ دارد و آخرین عنصر آن‌ها اندیس $(n-1)$ خواهد داشت که $n$
  همان اندازه‌ی آرایه‌ست. مثلاً اگر یک آرایه ۵تایی داشته باشیم، اندیس هایش
  از ۰ تا ۴ شماره‌گذاری می‌شوند.
\item
  \textbf{اندازهٔ ثابت:} وقتی یک آرایه را می‌سازیم، اندازه‌‌اش همان موقع
  مشخص می‌شود و بعد از آن دیگر نمی‌توانیم آن را بزرگ‌تر یا کوچک‌تر کنیم.
\item
  \textbf{نوع داده یکسان:} تمامی خانه‌های یک آرایه باید از یک نوع داده
  باشند، مثلاً اگر نوع دادهٔ آرایه \lr{\texttt{int}} باشد، نمی‌توانیم داخل
  آن \lr{\texttt{String}} یا \lr{\texttt{float}} بگذاریم. همه‌ٔ اعضای
  آرایه باید با هم هم‌نوع باشند. حالا بیایید ببینیم که آرایه در جاوا به
  چه صورت کار می‌کند:
\end{itemize}

\subsection{تعریف آرایه در جاوا}

حالا می‌خواهیم آرایه را تعریف کنیم، این کار را می‌توانیم به دو روش انجام
بدهیم:

\textbf{تعریف آرایه بدون مقدار دهی اولیه:} در این حالت فقط نوع داده‌ای که
می‌خواهیم ذخیره کنیم و اندازه‌ٔ آرایه مشخص می‌شود:

\begin{codebox}{java}
int[] numbers = new int[5];
\end{codebox}

در کد بالا، یک آرایهٔ پنج عضوی تعریف کردیم، ولی مقادیر اولیهٔ اعضای آن را
مشخص نکردیم. با این کار، اعضای آرایه مقدار دیفالت \lr{type} خود را
دراختیار می‌گیرند. مثلاً مقدار دیفالت یک متغیر \lr{\texttt{int}} که
مقداردهی نشده، صفر می‌باشد؛ به همین خاطر، اعضای آرایهٔ بالا همگی صفر
هستند. این موضوع را می‌توانید با چاپ این آرایه بررسی کنید:

\begin{codebox}{java}
for (int i = 0; i < numbers.length; i++) {  
    System.out.print(numbers[i] + ", ");  
}
\end{codebox}

خروجی کد بالا، به شکل زیر خواهد بود:

\begin{codebox}{text}
0, 0, 0, 0, 0,
\end{codebox}

مقادیر دیفالت typeهای عددی مثل \lr{\texttt{int}}، \lr{\texttt{float}} و
غیره در جاوا صفر است. مقدار دیفالت \lr{\texttt{boolean}} همان
\lr{\texttt{false}} بوده و مقدار دیفالت \lr{\texttt{char}} هم
\texttt{'\textbackslash n'} می‌باشد؛ اما مقدار دیفالت \lr{reference type}
ها همیشه \lr{\texttt{null}} است که در مباحث مربوط به \lr{OOP} راجع به
آن‌ها مطالعه خواهید کرد.

\textbf{تعریف آرایه با مقدار دهی اولیه:} در این حالت مقادیر آرایه همان
موقع که تعریف می‌کنیم مشخص می‌شوند.

\begin{codebox}{java}
int[] numbers = {1, 2, 3};
\end{codebox}

\subsection{دسترسی به عناصر آرایه}

بالاتر هم دیدید، در زبان \lr{C} هم دیدید! برای دسترسی به عناصر یک آرایه،
مثل کد زیر از \lr{index} آن‌ها استفاده می‌کنیم:

\begin{codebox}{java}
void main(String[] args) {
    int[] numbers = {1, 2, 3};
    int firstNumber = numbers[0];
    numbers[2] = 25;

    System.out.println(numbers);
}  
\end{codebox}

\subsection{پیمایش بر روی آرایه}

حالا بیایید تک تک عناصر آرایه را با استفاده از اندیس و یک حلقه
\lr{\texttt{for}} چاپ کنیم.

\begin{codebox}{java}
void main(String[] args) {
    int[] numbers = {1, 2, 3};
    for (int i = 0; i < numbers.length; i++) { 
    System.out.println("Index " + i + ": " + numbers[i]);
    }  
}  
\end{codebox}

ما، طول یک آرایه را با استفاده از \lr{\texttt{numbers.length}} بدست
می‌آوریم. به متغیرهایی مثل \lr{\texttt{length}}یا \lr{\texttt{variable}}
در جاوا \lr{field} های یک کلاس می‌گویند که از متغیرهای مربوط به آن کلاس
می‌باشد. با این متغیرها و انواع آن‌ها وقتی \lr{OOP} را یاد گرفتید بیشتر
آشنا خواهید شد. تا آن زمان، بیاید چندتا از متدها و فیلدهای پرکاربرد
آرایه را بررسی کنیم:

\subsection{متدها و فیلدهای کاربردی مرتبط با آرایه}

\begin{itemize}
\tightlist
\item
  متد \lr{\texttt{length}}
\end{itemize}

متد \lr{\texttt{length}} طول آرایه را به شما نشان می‌دهد:

\begin{codebox}{java}
int[] numbers = {1, 2, 3};
System.out.println("Array length: " + numbers.length); // Array length: 3
\end{codebox}

\begin{itemize}
\tightlist
\item
  متد \lr{\texttt{System.arraycopy}}
\end{itemize}

برای کپی کردن آرایه‌ها می‌توانید از متد \lr{\texttt{System.arraycopy()}}
استفاده کنید، می‌توانید با استفاده از کد زیر، که آرایهٔ‌
\lr{\texttt{numbers}} را داخل \lr{\texttt{copiedArray}} کپی می‌کند، این
متد را بررسی کنید:

\begin{codebox}{java}
int[] numbers = {1, 2, 3};
int[] copiedArray = new int[numbers.length];  
  
System.arraycopy(numbers, 0, copiedArray, 0, numbers.length);  
  
System.out.print("Copied array: ");
for (int i = 0; i < copiedArray.length; i++) {  
    System.out.print(copiedArray[i] + " ");  
}
\end{codebox}

خروجی این کد، به شکل زیر می‌باشد:

\begin{codebox}{text}
Copied array: 1 2 3
\end{codebox}

\begin{itemize}
\tightlist
\item
  متد \lr{\texttt{Arrays.sort}}
\end{itemize}

با استفاده از متد \lr{\texttt{Arrays.sort}}، می‌توانید آرایه‌ها را سورت
کنید. قبل از استفاده از آن، توجه کنید که حتماً پکیج \lr{(package)}
\lr{\texttt{java.util.Arrays}} را \lr{\texttt{import}} کرده باشید:

\begin{codebox}{java}
import java.util.Arrays;   

void main(String[] args) { 
    int[] numbers = {5, 3, 8, 1, 2};
    System.out.println("Original array: " + Arrays.toString(numbers)); 

    Arrays.sort(numbers); 
    System.out.println("Sorted array: " + Arrays.toString(numbers));
}  
\end{codebox}

حال که آرایهٔ یک بعدی را یاد گرفته ایم، یک مرحله بالاتر برویم و در مورد
آرایهٔ چند بعدی صحبت کنیم.

\section{آرایه‌های چندبعدی}

آرایه‌های چندبعدی برای ذخیره و مدیریت داده‌های پیچیده‌تر به کار می‌روند.
تصویر پایین هم یک مثال خوب از آرایهٔ دو بعدی را نشان می‌دهد:

\begin{figure}[H]
\centering
\includegraphics[width=0.8\textwidth]{images/multi-dimensional-array.png}
\end{figure}

\begin{figure}[H]
\centering
\includegraphics[width=0.8\textwidth]{images/multi-dimensional-array2.png}
\end{figure}

در تصویر اول و سمت چپ تصویر دوم، شما یک آرایهٔ دوبعدی می‌بینید که هر کدام
از عناصر آرایهٔ بیرونی، خود یک آرایه هستند. برای راحت‌تر فهمیدن، می‌توانید
آرایهٔ دو بعدی را مثل یک ماتریس در نظر بگیرید. درکنار آرایهٔ دوبعدی، شما
آرایهٔ سه‌بعدی را می‌بینید که درواقع آرایه‌ای از آرایه‌های دوبعدی است؛ یعنی
هرعنصر آن، خود یک آرایهٔ دوبعدی است.

\subsection{تعریف آرایه چند بعدی در جاوا}

در ابتدا، نحوهٔ تعریف آرایه‌های دو بعدی و چند بعدی را تعریف می‌کنیم که مثل
آرایهٔ یک‌بعدی، به دو روش مختلف می‌شود انجامش داد.

\textbf{تعریف آرایه دو بعدی و چند بعدی بدون مقدار دهی اولیه:} مثل قبل،
می‌توانید بدون مقداردهی اولیه آرایه تعریف کنید و باز هم خانه‌های آرایه‌ها
با مقدار دیفالت \lr{type} شان پر می‌شوند.

\begin{codebox}{java}
// 2D array with 3 rows and 4 columns
int[][] matrix = new int[3][4];

// 3D array with 3 layers, each containing 3 rows and 3 columns
int[][][] cube = new int[3][3][3];  
  
System.out.println(matrix[0][0]); // 0
System.out.println(cube[0][0][0]); // 0
\end{codebox}

\textbf{تعریف آرایه دو بعدی و چند بعدی با مقدار دهی اولیه:} این روش هم،
مشابه آرایه‌های یک بعدی هست:

\begin{codebox}{java}
int[][] matrix = {  
        {1, 2, 3},  
        {4, 5, 6},  
        {7, 8, 9}  
};

int[][][] cube = {  
        {  
                {1, 2, 3},  
                {4, 5, 6}  
        },  
        {  
                {7, 8, 9},  
                {10, 11, 12}  
        }  
};
\end{codebox}

\subsection{دسترسی به عناصر آرایه‌های چند بعدی}

مجدداً، با استفاده از اندیس‌ها مثل زبان \lr{C} ، می‌توانیم به عناصر
آرایه‌های چندبعدی دست پیدا کنیم:

\begin{codebox}{java}
int[][] matrix = {  
        {1, 2, 3},  
        {4, 5, 6},  
        {7, 8, 9}  
};  
  
int value1 = matrix[1][2];  
matrix[0][1] = 42;
\end{codebox}

\subsection{پیمایش آرایه‌های چند بعدی}

حالا بیایید تک تک عناصر آرایه را با استفاده از اندیس و یک حلقهٔ
\lr{\texttt{for}} چاپ کنیم.

\begin{codebox}{java}
void main(String[] args) { 
    int[][] matrix = {  
        {1, 2, 3},
        {4, 5, 6},
        {7, 8, 9}  
    };

    System.out.println("Matrix contents:");  
    for (int i = 0; i < matrix.length; i++) {
        for (int j = 0; j < matrix[i].length; j++) {  
            System.out.print(matrix[i][j] + " ");
        System.out.println();
        }  
    }  
}
\end{codebox}

تنها چیزی که داخل این کد، نسبت به کدی که برای آرایهٔ یک‌بعدی زدیم جدید
است، متغیر \lr{\texttt{j}} است که از صفر شروع می‌شود و تا
\lr{\texttt{matrix[i].length}} پیش می‌رود.
\lr{\texttt{matrix[i].length}}، در واقع طول $i$-امین آرایه‌ای است که توی
\lr{matrix} هست.

\section{\texttt{ArrayList}}

تصور کنید که یک ساختمان داده داشته باشیم که شبیه آرایه باشد، ولی بتوانیم
اندازه‌اش را هر موقع که خواستیم تغییر بدهیم. این ویژگی را \lr{ArrayList}
به ما می‌دهد. ArrayList، سایز متغیری دارد که آن را بر اساس مقادیر داخلش
مشخص می‌کند و استفاده از آن خیلی شبیه به یک \lr{Array} معمولی است. حالا
می‌خواهیم یک \lr{Array List} بسازیم و سپس مهم‌ترین متدهایی که دارد را
توضیح بدهیم.

\subsection{\lr{reference type} های معادل \lr{primitive type} ها در جاوا}

برای هر \lr{primitive type} توی جاوا، مثل \lr{\texttt{int}}،
\lr{\texttt{boolean}}، \lr{\texttt{char}} و امثال آن‌ها، یک معادل از جنس
\lr{reference type} وجود دارد. وقتی که با مفهوم \lr{object} ها بیشتر
آشنا بشویم، این موضوع را مجددا بررسی می‌کنیم، ولی چون برای
\lr{\texttt{ArrayList}} به آن‌ها نیاز داریم، لازم است که یک نگاه مختصر به
آن‌ها بیندازیم.

کد زیر، یک متغیر \lr{\texttt{int}} ساده تعریف می‌کند:

\begin{codebox}{java}
int num = 10;
\end{codebox}

ما می‌توانیم این متغیر را، به جای \lr{\texttt{int}}، از جنس
\lr{\texttt{Integer}} تعریف کنیم:

\begin{codebox}{java}
Integer num = 0;
\end{codebox}

تنها تفاوت این دو، این است که \lr{\texttt{Integer}}، \lr{reference type}
ای است که معادل \lr{\texttt{int}} می‌باشد. کامپایلر شما حتی به شما این
اجازه را می‌دهد که با متغیری از جنس \lr{\texttt{Integer}} مثل یک
\lr{\texttt{int}} برخورد کنید:

\begin{codebox}{java}
int a = 10;  
Integer b = a;  
int c = a + b;  
  
System.out.println(a);  
System.out.println(b);
\end{codebox}

در مورد این که چرا وقتی \lr{\texttt{int}} را داریم، به
\lr{\texttt{Integer}} هم نیاز داریم بعدا توضیح خواهیم داد. برای الآن
کافی است بدانید که این \lr{reference type} ها وجود دارند و تا حد خیلی
خوبی هم معادل همتای \lr{primitive type} شان هستند. فهرست کامل آن‌ها در
جدول زیر آمده است:

\begin{table}[H]
\centering
\setLTR
\begin{tabular}{ll}
\hline
\textbf{\lr{Primitive Type}} & \textbf{\lr{Reference Type (Wrapper Class)}} \\
\hline
\texttt{boolean} & \texttt{Boolean} \\
\texttt{byte} & \texttt{Byte} \\
\texttt{char} & \texttt{Character} \\
\texttt{short} & \texttt{Short} \\
\texttt{int} & \texttt{Integer} \\
\texttt{long} & \texttt{Long} \\
\texttt{float} & \texttt{Float} \\
\texttt{double} & \texttt{Double} \\
\hline
\end{tabular}
\end{table}

در \lr{\texttt{ArrayList}} ها، ما از این \lr{reference type} ها استفاده
می‌کنیم.

\subsection{ساخت اولین \lr{ArrayList}}

اول از همه باید \lr{package} مربوط به \lr{\texttt{ArrayList}} را
\lr{\texttt{import}} کنیم.

\begin{codebox}{java}
import java.util.*;
\end{codebox}

سپس، با تکه کد زیر می‌توانیم یک \lr{\texttt{ArrayList}} از
\lr{\texttt{Float}}ها و \lr{\texttt{Integer}}ها تعریف کنیم:

\begin{codebox}{java}
void main(String[] args) {  
    ArrayList<Integer> intNumbers = new ArrayList<Integer>();  
    ArrayList<Float> floatNumbers = new ArrayList<Float>();  
}  
\end{codebox}

دقت کنید که شما نمی‌توانید \lr{\texttt{ArrayList}}ای از جنس
\lr{primitive type} تعریف کنید. به همین خاطر است که ما از
\lr{\texttt{Integer}} و \lr{\texttt{Float}} استفاده کردیم. چیزی که بین
\lr{\texttt{<}} و \lr{\texttt{>}} جلوی \lr{ArrayList} می‌آید، جنس چیزی
هست که داخل \lr{ArrayList} نگه می‌دارید.

شما می‌توانید تایپ کامل \lr{\texttt{ArrayList}} را مقابل کلیدواژهٔ
\lr{\texttt{new}} ننویسید و داخل \lr{\texttt{<>}} را خالی بگذارید، چون
که قبل از اسم متغیر، تایپ آن را کامل مشخص کردید:

\begin{codebox}{java}
ArrayList<Integer> numbers = new ArrayList<>();
\end{codebox}

یا این که به کل از \lr{\texttt{var}} استفاده کنید:

\begin{codebox}{java}
var numbers = new ArrayList<Integer>();
\end{codebox}

\subsection{متدهای \lr{ArrayList}}

\begin{itemize}
\tightlist
\item
  متد \lr{\texttt{add}}
\end{itemize}

برای اضافه کردن عناصر به \lr{\texttt{ArrayList}} از این متد استفاده
می‌کنیم. می‌توانید به دو شکل زیر به یک \lr{\texttt{ArrayList}} عنصر اضافه
کنید:

\textbf{اضافه کردن به آخر لیست:} کد زیر به خوبی به شما نشان می‌دهد که
چگونه می‌توانید از ساده‌ترین نوع \lr{\texttt{add}} استفاده کنید.

\begin{codebox}{java}
import java.util.ArrayList;  
  
void main(String[] args) {
    ArrayList<Integer> numbers = new ArrayList<Integer>();

    numbers.add(10);
    numbers.add(20);
    numbers.add(30);
    numbers.add(40);
        
    System.out.println(numbers); // Output: [10, 20, 30, 40]    
}  
\end{codebox}

همچنین می‌توانید، به جای این که از چندتا \lr{\texttt{add}} استفاده کنید
تا مقداردهی اولیهٔ آرایه‌تان را انجام بدهید، آن را به این شکل تعریف کنید:

\begin{codebox}{java}
ArrayList<Integer> numbers = new ArrayList<>(Arrays.asList(1, 3000, 40));  
System.out.println(numbers); // Output: [1, 3000, 40]
\end{codebox}

با این کار، لازم نیست تا برنامه‌تان را پر از \lr{\texttt{add}} بکنید.

\textbf{اضافه کردن در یک ایندکس مشخص از لیست:} می‌توانید با استفاده از
متد زیر، به یک جای مشخص از \lr{\texttt{ArrayList}} یک عنصر را اضافه
کنید.

\begin{codebox}{java}
ArrayList<Integer> numbers = new ArrayList<>(Arrays.asList(1, 3000, 40));  
System.out.println(numbers); // Output: [1, 3000, 40]

numbers.add(1, 15);  
System.out.println(numbers); // Output: [1, 15, 3000, 40]
\end{codebox}

\begin{itemize}
\tightlist
\item
  متد \lr{\texttt{remove}}
\end{itemize}

می‌توانیم با ورودی دادن \lr{index} عنصر دلخواهمان، آن را از آرایه پاک
کنیم:

\begin{codebox}{java}
ArrayList<Character> letters = new ArrayList<>(Arrays.asList('A', 'B', 'C', 'D', 'E', 'F'));  
letters.remove(2); // Removes the third element of letters

System.out.println(letters); // Output: [A, B, D, E, F]
\end{codebox}

\begin{itemize}
\tightlist
\item
  متد \lr{\texttt{get}}
\end{itemize}

این متد برای دسترسی به یک عنصر بر اساس \lr{index} آن استفاده می‌شود و
شبیه \lr{\texttt{arr[i]}}ای است که در \lr{Array} دیده بودیم:

\begin{codebox}{java}
var specialNumbers = new ArrayList<Double>(Arrays.asList(3.14, 2.71));
var pi = specialNumbers.get(0);  
  
System.out.println(pi); // Output: 3.14
\end{codebox}

\begin{itemize}
\tightlist
\item
  متد \lr{\texttt{size}}
\end{itemize}

این متد تعداد عناصر موجود در لیست را برمی‌گرداند.

\begin{codebox}{java}
var names = new ArrayList<String>(Arrays.asList("Negin", "Paria", "Kiana"));

for (int i = 0; i < names.size(); i++) {  
    System.out.println(i + ". " + names.get(i));  
}
\end{codebox}

\begin{itemize}
\tightlist
\item
  متد \lr{\texttt{set}}
\end{itemize}

برای تغییر مقدار یک عنصر در یک ایندکس خاص از این متد استفاده می‌شود.

\begin{codebox}{java}
var names = new ArrayList<String>(Arrays.asList("Negin", "Paria", "Kiana"));  
System.out.println(names); // Output: [Negin, Paria, Kiana]  
  
names.set(1, "Dorsa");  
System.out.println(names);// Output: [Negin, Dorsa, Kiana]
\end{codebox}

\section{کلاس استرینگ در جاوا}

اکنون می‌خواهیم با یکی از مهم‌ترین کلاس‌های جاوا آشنا شویم. استرینگ در جاوا
آرایه‌ای از کارکترهاست که به صورت زیر و با \lr{\texttt{""}} تعریف می‌شود.

\begin{codebox}{java}
void main(String[] args) {  
    String str = "write your string here";
}  
\end{codebox}

\subsection{متدهای کلاس استرینگ}

اکنون در کد خود یک رشته به صورت گفته شده تعریف کنید. در زیر چندین متد
مهم از کلاس استرینگ گفته شده است. آن‌ها را بر روی رشته ی خود اجرا کنید و
کاربردهای گوناگون آنها را ببینید.

\begin{itemize}
\tightlist
\item
  متد \lr{\texttt{charAt}}
\end{itemize}

فرض کنید می‌خواهیم به بخش‌های گوناگون یک رشته دسترسی یابیم. متد
\lr{\texttt{charAt()}} برای انجام این کار به ما کمک می‌کند. این متد
کارکتر موجود در ایندکس مشخص شده از رشته را برمی‌گرداند.

\begin{codebox}{java}
String str = "write your string here";  
str.charAt(3); //returns t
\end{codebox}

\begin{itemize}
\tightlist
\item
  متد \lr{\texttt{toCharArray}}
\end{itemize}

این متد در هنگامی استفاده می‌شود که نیاز داریم روی هر کارکتر یک رشته
عملیات خاصی انجام دهیم. این متد یک رشته را به آرایه‌ای از کارکترها تبدیل
می‌کند.

\begin{codebox}{java}
String str = "Hello";  
char[] charArray = str.toCharArray();  
for (int i = 0; i < charArray.length-1; i++) {  
    System.out.print(charArray[i] + ",");
}  
System.out.print(charArray[charArray.length-1]);
\end{codebox}

خروجی این کد \lr{\texttt{H,e,l,l,o}} است.

\begin{itemize}
\tightlist
\item
  متد \lr{\texttt{concat}}
\end{itemize}

به انتهای رشته، رشته‌ٔ مشخص شده را اضافه می‌کند.

\begin{codebox}{java}
String str = "Hello";  
String name = " Negin";  
str.concat(name); //returns Hello Negin
\end{codebox}

یک روش دیگر نیز برای افزودن رشته‌ها به یکدیگر وجود دارد:

\begin{codebox}{java}
String str = "Hello";  
String name = " Paria";  
String result = str + name;  
System.out.println(result);
\end{codebox}

خروجی این کد \lr{\texttt{Hello Paria}} است.

\begin{itemize}
\tightlist
\item
  متد \lr{\texttt{contains}}
\end{itemize}

خروجی این متد \lr{\texttt{Boolean}} است. بررسی می‌کند که آیا در درون
رشته، رشته‌ٔ مشخص‌شده وجود دارد یا خیر. اگر وجود داشته باشد
\lr{\texttt{true}} و در غیر این صورت \lr{\texttt{false}} را برمی‌گرداند.

\begin{codebox}{java}
String str = "Gharakhani’s house";  
str.contains("house");
\end{codebox}

\begin{itemize}
\tightlist
\item
  متد \lr{\texttt{endsWith}}
\end{itemize}

متد دیگری که بررسی می‌کنیم متد \lr{\texttt{endsWith()}} است. این متد
بررسی می‌کند که آیا رشته با عبارت مشخص‌شده پایان یافته است یا خیر.

\begin{codebox}{java}
String str = "write your string here";  
str.endsWith("here");
\end{codebox}

\begin{itemize}
\tightlist
\item
  متد \lr{\texttt{startsWith}}
\end{itemize}

متد \lr{\texttt{startsWith()}} مشابه متد قبلی است. بررسی می‌کند که آیا
رشته با عبارت مشخص‌شده شروع یافته است یا خیر.

مثلاً فرض کنید شما در برنامه‌‌ از کاربران شماره تلفنشان را می‌خواهید.
میتوانید با این متد بررسی کنید که شماره فرمت درستی دارد و با
\lr{\texttt{091}} شروع شده است یا خیر.

\begin{itemize}
\tightlist
\item
  متد \lr{\texttt{equals}}
\end{itemize}

دو رشته را باهم مقایسه می‌کند. اگر برابر بودند \lr{\texttt{true}} را
برمی‌گرداند و در غیر این صورت \lr{\texttt{false}} را برمی‌گرداند.

\begin{codebox}{java}
String str = "AmirKabir University";  
String sample= "University";  
str.equals(sample); //returns false  
str.equals("AmirKabir University"); //returns true
\end{codebox}

\begin{itemize}
\tightlist
\item
  متد \lr{\texttt{indexOf}}
\end{itemize}

اولین جایگاه کارکتر مشخص‌شده در رشته را برمی‌گرداند. اگر کارکتر در استرینگ
وجود نداشت $-1$ برمی‌گرداند.

\begin{codebox}{java}
String str = "write your string here";  
str.indexOf('t'); //returns 3  
str.indexOf('f'); //returns -1
\end{codebox}

\begin{itemize}
\tightlist
\item
  متد \lr{\texttt{length}}
\end{itemize}

این متد بسیار کاربردی نیز، طول استرینگ داده شده را برمی‌گرداند.

\begin{codebox}{java}
String str = "write your string here";  
str.length(); //returns 22
\end{codebox}

\begin{itemize}
\tightlist
\item
  متد \lr{\texttt{split}}
\end{itemize}

این متد رشته را با استفاده از کارکتر داده شده به بخش‌های جداگانه تقسیم
می‌کند و نتیجه را به صورت یک آرایه بازمی‌گرداند.

\begin{codebox}{java}
String str = "pizza,pasta,burger";  
String[] arr = str.split(",");  
System.out.println(arr[1]);
\end{codebox}

خروجی این کد \lr{\texttt{pasta}} است.

میتوانید کلماتی که میان آنها فاصله است را نیز با استفاده از این متد جدا
کنید.

\begin{codebox}{java}
String str = "pizza pasta burger";  
String[] arr = str.split(" ");  
System.out.println(arr[1]);
\end{codebox}

خروجی این کد نیز \lr{\texttt{pasta}}است.

\begin{itemize}
\tightlist
\item
  متد \lr{\texttt{substring}}
\end{itemize}

در پایان با یکی از کاربردی ترین متد های کلاس استرینگ آشنا می‌شویم. این
متد از ایندکس اولیه‌ٔ داده شده (\lr{\texttt{begIndex}}) تا قبل از ایندکس
پایانی (\lr{\texttt{endIndex}}) رشته را باز می‌گرداند. میتوانید
\lr{\texttt{endIndex}} را نگذارید و آنگاه تا اخر رشته برگردانده میشود.
اکنون چند مثال از این متد می‌بینیم:

\begin{codebox}{java}
String str = "just keep swimming";  
String result = str.substring(5,9);  
System.out.println(result);
\end{codebox}

خروجی این کد \lr{\texttt{keep}} است.

اما اگر ایندکس پایانی را مانند مثال زیر نگذاریم:

\begin{codebox}{java}
String str = "just keep swimming";  
String result = str.substring(5);  
System.out.println(result);
\end{codebox}

در اینجا تا پایان رشته گرفته شده و \lr{\texttt{keep swimming}} در خروجی
چاپ می‌شود.

حال با مهم ترین متدهای کلاس استرینگ آشنا شدید. این کلاس متدهای بسیار
زیاد و گوناگونی دارد که در
\href{https://www.w3schools.com/java/java_ref_string.asp}{این صفحه}
می‌توانید با برخی دیگر از این متدها آشنا شوید.

حال بیایید یک سوال ساده را با استفاده از متدهایی که یاد گرفتیم حل کنیم.

\subsection{سوال \texttt{Wow!}}

نگین که از کد زدن خسته شده است، به تازگی به رشته‌ٔ مهندسی برق علاقه پیدا
کرده است. به همین دلیل تصمیم گرفته است تا درباره‌ٔ این رشته تحقیق کند. او
به افراد مختلف مراجعه می‌کند و هرکدام یک مقدار اطلاعات به او می‌دهند. او
به اندازه‌ی مقدار اطلاعاتی که از هر فرد می‌گیرد، متعجب می‌شود. مثلا اگر یک
عدد اطلاعات بگیرد می‌گوید \lr{\texttt{Wow!}}، اگر سه تا اطلاعات بگیرد
میگوید \lr{\texttt{Wooow!}} و به همین شکل تعداد \lr{\texttt{o}} ها زیاد
میشود. حال اگر یک نفر به اندازه‌ی \lr{\texttt{n}} اطلاعات به نگین بدهد،
ما باید انتظار چه کلمه‌ای را از او داشته باشیم؟

سعی کنید سوال بالا را با متدهایی که آموزش دیده‌اید حل کنید. برای مثال
می‌توانید در هر مرحله با استفاده از متد \lr{\texttt{substring()}} بخش
\lr{\texttt{W}} ابتدایی را جدا کرده و در هر مرحله با استفاده از متد
\lr{\texttt{concat()}} یک \lr{\texttt{o}} به آن بیافزایید. این کار را
\lr{\texttt{n-1}} بار تکرار کنید. رشته‌ی نهایی پاسخ ما است.

\section{\lr{HashMap}}

فرض کنید می‌خواهیم نام دانشجویان و شماره دانشجویی آن‌ها را در یک
\lr{Data Structure} ذخیره کنیم. می‌خواهیم به سرعت و راحتی به آن‌ها دسترسی
داشته باشیم و در عین حال دانشجویان را از هم تفکیک کنیم. باتوجه به آن که
هر دانشجو یک شماره دانشجویی یکتا دارد، ما از یک \lr{Data Structure} که
از این ویژگی بهره می‌برد استفاده می‌کنیم. کلاس \lr{HashMap} یکی از
نمونه‌های خوب این گونه ساختمان داده‌هاست.

در این بخش، هدف ما آشنایی شما با مفهوم و کاربردهای کلاس دیکشنری است.
دیکشنری یک نوع ساختار داده است که برای ذخیره کلید \lr{(key)} و مقدار
\lr{(value)} استفاده می‌شود. برای مثال یک دفترچه تلفن را در نظر بگیرید.
اسم افراد نقش \lr{\texttt{key}} و شماره تلفنشان نقش \lr{\texttt{value}}
را دارد. مثلا به برنامه می‌گویید که کلید \lr{\texttt{Paria}} مقدار
\lr{\texttt{09123456789}} دارد. کلیدها یکتا \lr{(unique)} هستند. مثلا
شما دو کلید \lr{\texttt{Paria}} نمی‌توانید داشته باشید.

\subsection{تعریف \lr{HashMap}}

در جاوا، به شکل زیر می‌توانید یک HashMap، که جنس \lr{\texttt{key}} های آن
\lr{\texttt{String}} و جنس \lr{\texttt{value}} هایش
\lr{\texttt{Integer}} است تعریف کنید:

\begin{codebox}{java}
import java.util.HashMap;

void main(String[] args) {  
    HashMap<String,Integer> studentId = new HashMap<>();
}  
\end{codebox}

حال، می‌توانید با استفاده از متد \lr{\texttt{put}}، به این
\lr{\texttt{HashMap}} مقادیر دلخواه خود را اضافه کنید:

\begin{codebox}{java}
studentId.put("Negin", 13001);  
studentId.put("Paria", 13002);
\end{codebox}

بعد از اجرای کد بالا، هش‌مپ ما، عدد \lr{\texttt{13001}} را برای کلید
\lr{\texttt{Negin}} و عدد \lr{\texttt{13002}} را برای
\lr{\texttt{Paria}} ذخیره کرده است. ما می‌توانیم با استفاده از متد
\lr{\texttt{get}} و ورودی دادن کلید دلخواهمان، مقدار متناظر با آن کلید
را به دست بیاوریم:

\begin{codebox}{java}
var neginStudentId = studentId.get("Negin");  
System.out.println(neginStudentId); // Output: 13001
\end{codebox}

اگر کلیدی در \lr{HashMap} وجود نداشته باشد و ما آن را به عنوان ورودی به
متد \lr{\texttt{get}} بدهیم، خروجی این متد برای ما \lr{\texttt{null}}
است. \lr{\texttt{null}} ، مقدار خاصی برای \lr{reference type} هاست که
نشان می‌دهد که آن‌ها هنوز وجود ندارند و با آن در \lr{OOP} بیشتر آشنا
می‌شویم:

\begin{codebox}{java}
System.out.println(studentId.get("Yasin")); // Output: null
\end{codebox}

برای این که چک کنیم که هش‌مپ‌مان کلیدی را دارد یا نه، از
\lr{\texttt{containsKey}} استفاده می‌کنیم:

\begin{codebox}{java}
if (studentId.containsKey("Paria")) {  
    System.out.println("Paria: " + studentId.get("Paria"));  
} else {  
    System.out.println("Cannot find Paria");  
}
\end{codebox}

\subsection{کاربردهای \lr{HashMap}}

حال بیایید چند مثال و کاربرد از کلاس دیکشنری ببینیم تا به درک بهتری از
آن برسیم.

\begin{itemize}
\tightlist
\item
  پیدا کردن سریع داده‌ها: میتوانید با دادن \lr{\texttt{key}} خیلی سریع
  \lr{\texttt{value}} آن را پیدا کنید. در درس ساختمان داده خواهید دید که
  این کار، بسیار سریع تر از گشتن در یک آرایه است:
\end{itemize}

\begin{codebox}{java}
HashMap<String,Integer> studentId = new HashMap<>();  
  
studentId.put("Negin",13001);  
studentId.put("Paria",13002);

var neginStudentId = studentId.get("Negin"); // neginStudentId = 13001
\end{codebox}

\begin{itemize}
\tightlist
\item
  شمارش تعداد دفعات تکرار کلمات: با هر بار دیدن کلمه در متن
  \lr{\texttt{value}} تغییر داده شود و یک عدد به آن اضافه شود.
\end{itemize}

\begin{codebox}{java}
HashMap<String,Integer> studentCredit = new HashMap<>();  
  
studentCredit.put("Negin",0);  
studentCredit.put("Paria",0);  
  
String creditsList = "Paria,Negin,Paria";  
  
String[] gotCreditStudent = creditsList.split(",");
for (String studentName : gotCreditStudent){
    int temp = studentCredit.get(studentName);  
  
    studentCredit.put(studentName, temp + 1);  
}  
  
System.out.println("Paria's Credit: " + studentCredit.get("Paria"));  
System.out.println("Negin's Credit: " + studentCredit.get("Negin"));
\end{codebox}

حلقهٔ \lr{\texttt{for}} استفاده شده در اینجا به صورت
\lr{enhanced for loop} است که اگر اکنون با آن آشنایی ندارید پس از به
پایان رساندن \lr{Document} برگردید و این کد را دوباره مطالعه کنید.

\begin{itemize}
\tightlist
\item
  نگهداری ارتباط بین دو مجموعه داده: برای مثال \lr{\texttt{key}} نام
  دانش‌آموز و \lr{\texttt{value}} مجموعه نمراتش باشد.
\item
  گروه‌بندی داده‌ها: برای مثال \lr{\texttt{key}} الکترونیک و
  \lr{\texttt{value}} لیست محصولات الکترونیک (لپ تاپ، تلفن همراه و
  \ldots) باشد.
\end{itemize}

\section{\lr{enhanced for loop}}

حالا فرض کنید وقتی داریم یک آرایه یا هر ساختمان دادهٔ دیگه‌ای را پیمایش
می‌کنیم و فقط به مقدار آن عنصر نیاز داریم و اصلاً کاری به ایندکسش نداریم.
یعنی تمرکز ما بیشتر روی خود مقدار عنصر است تا ایندکسش. در این مواقع
می‌توانیم از \lr{enhanced for loop} استفاده کنیم.

در واقع، دستور \lr{enhanced for loop} همان حلقه \lr{\texttt{for}} معمولی
است که تعدادی تغییرات در ساختار آن ایجاد کردیم. یکی از مهم‌ترین تفاوت‌هایش
این است که در \lr{enhanced for loop} تمرکز بیشتر روی مقدار و ارزش هر
عنصر است. این دستور برای پیمایش آرایه‌ها، \lr{\texttt{ArrayList}} و
\ldots{} استفاده می‌شود.

بگذارید با یک مثال بهتر توضیح دهیم:

\begin{codebox}{java}
void main(String[] args) {
    // An array of fruits
    String[] fruits = {"Apple", "Banana", "Orange"};

    // Iterate over the array using for-each loop
    for (String fruit : fruits) {
        // Print each fruit
        System.out.println(fruit);
    }  
}  
\end{codebox}

در اینجا متغیر \lr{\texttt{fruits}} آرایه ای از رشته‌ها است که در آن سه
عنصر وجود دارد. این حلقه \lr{\texttt{for}} جوری کار می‌کند که از اولین
عنصر شروع می‌کند و یکی‌یکی تمام مقدارهای داخل آرایه را می‌گیرد. در هر دور
از حلقه، مقدار \lr{\texttt{fruit}} برابر با یکی از عناصر آرایه می‌شود
(مثلاً در دور اول \lr{\texttt{Apple}} ، دور دوم \lr{\texttt{Banana}} و
دور سوم \lr{\texttt{Orange}} ). بعد، مقدار \lr{\texttt{fruit}} چاپ
می‌شود. همان‌طور که دیدیم، توی این مدل \lr{\texttt{for}} که به آن
\lr{enhanced for loop} هم می‌گویند، بیشتر روی مقدار عنصر تمرکز داریم تا
ایندکسش.
	
	
	
	
%%%%%%%%%%%%%%%%%%%%%%%%%%%%%%%%%%%%%%%%%%%%%%%%%%%%%%%%%%%%
\end{document}